\documentclass[a4paper,11pt,american]{article}
\usepackage[a4paper,margin=1.2in]{geometry}

\usepackage{amsmath,amssymb,amsfonts,mathrsfs} %important math packages
\usepackage{amsthm} % Package for theorem and definition enviroments
\usepackage{babel} % Language control, for hyphenation etc
\usepackage{bbm}
\usepackage{booktabs} % Better looking tables
\usepackage{caption, subcaption, graphicx}
\usepackage{comment}
\usepackage[babel]{csquotes} % Better looking quotes
\usepackage{enumerate} % package for making different lists
\usepackage[shortlabels]{enumitem}
\usepackage{float}
\usepackage[T1]{fontenc} % Encoding of fonts
\usepackage[hidelinks]{hyperref} %enable links for reference
\usepackage{lmodern} % Latin modern font - needed for fontenc
\usepackage[utf8]{inputenc} % Encoding of input text
\usepackage[kerning]{microtype} % Better looking text
\usepackage{mathpazo}
\usepackage{tcolorbox}


% This data must be adjusted per sheet:
\newcommand{\GROUP}{'Group number'}%replace this with the number of your group
\newcommand{\EXERCISES}{"Exercise number"}%replace this with the number of your exercise


\newcommand{\COURSE}{5330 Assignment \EXERCISES}
\newcommand{\TUTOR}{Zhaoqin Zhu}
\newcommand{\STUDENTA}{John Doe}
\newcommand{\STUDENTB}{Jane Doe}
\newcommand{\STUDENTC}{Max Mustermann}
\newcommand{\DEADLINE}{XX.0X.2019}

\usepackage{fancyhdr}%please don't change this display style for the homework, but later on you can use this chunk to determine what to display on the head and foot of a page
\pagestyle {fancy}
\fancyhead[L]{Group: \GROUP}
\fancyhead[C]{}
\fancyhead[R]{\COURSE}

\fancyfoot[L]{}
\fancyfoot[C]{}
\fancyfoot[R]{\thepage}


% define command for convenience, notice that these commands are user defined, so you cannot google for an answer.
%format: \newcommand{'your new command in the file, cannot be googled'}{'what it is supposed to do, can be googled'}
\newcommand{\reals}{\mathbb{R}}
\newcommand{\xx}{\mathbf{x}}
\newcommand{\pp}{\mathbf{p}}
\newtheorem{theorem}{Theorem}

\title{Answers to Advanced microeconomics problem set \EXERCISES}
\author{Your name here \thanks{Your email here}} % change your name and email here
\date{\today}
\setlength\parindent{0pt}

\begin{document}

\maketitle

I really recommend using the website "overleaf" for this work, which not only allows you write latex efficiently, but also allows many people to write on the same document! 

To use this template, simply create a blank project in latex, and copy paste every word from this tex file into the template. The .tex file can also be opened by txt reader in Windows system.
\section{Intro to latex}%you can start typing here
%\begin{enumerate} allows you to organize answers with points efficiently
\begin{enumerate}[(a)]
    \item 
    In most cases, the output starts from the line "begin\{ documents \}" and ends before the line "end\{ documents \}". Before the main text, people write lines to specify which package they want to use, what layout they want to have, and what personal commands they want to define.
    Latex is usually powered by "environments", which stars with "begin\{environment\_1\}"\footnote{Also, notice how I typed the symbol "\_" in text} and ends with "end\{environment\_1\}". They can be nested into one another, but each "begin" must corresponds to an "end", and that is where beginners get things wrong.

    \item 
    The basic expression for math is $x+1=2$, if you want to stress an equation, use
    \begin{equation}
        x+1 = 2
    \end{equation}
    Use $\{$ and $\}$ to type the bracket symbol\footnote{By now you should realize that $\textbackslash$ can be added to escape certain symbols in latex}, use $x^2_1$ to type superscripts and subscripts.
    To start a new line, use \\
    The commands can be stacked up for many equations and lines.

    \item 
    The command above is to start a new item, within "enumerate" environment. Notice the "newcommand" commands in the template, they usually bring unique personally defined commands that cannot be googled.

    \item 
    If you cannot get a document, look at the errors and warnings, google them!

    \item 
    use \newpage to start a new page, use 
    \begin{figure}
        \centering
        \includegraphics[scale=1]{How to draw a horse meme.png}%scale= controls how big your figure is in the text
        \caption{Step 4 is really good enough for beginners}
        \label{fig:enter-label}%you can change the label to distinguish different figures and equations, just ref the correct item when you need it
    \end{figure}
    to add a figure. Since your figure may fly around, use figure \ref{fig:enter-label} to make a reference. If you are using overleaf, the original figure should be uploaded to overleaf. The figure itself can be a photo taken from a phone, or drawings from "AutoDesk Sketchbook" or "Windows Ink Whiteboard"
    
\end{enumerate}

\section{Exercise 1.17}%you can start typing here

\section{Exercise 1.20}

\section{Exercise 1.28, not in the homework, only for reference}
This is to help you know how to type different equations and symbols in latex. You can also google for answer.
\subsection{Original question}
In the proof of Theorem 1.4 we use the fact that if $u(\cdot)$ is quasiconcave and differentiable at $\xx$ and $u(\mathbf{y}) \geq u(\xx)$, then $\nabla u(\xx) \cdot (\mathbf{y} - \xx) \geq 0$. Prove this fact in the following two steps.
\begin{enumerate}[label = \alph*)]
\item Prove that if $u(\mathbf{y}) \geq u(\xx)$, the quasiconcavity of $u(\cdot)$ and its differentiability at $\xx$ imply that the derivative of $u((1 - t) \xx + t \mathbf{y})$ with respect to $t$ must be non-negative at $t = 0$.
\item Compute the derivative of $u((1 - t) \xx + t \mathbf{y})$ with respect to $t$ evaluated at $t = 0$ and show that it is $\nabla u(\xx) \cdot (\mathbf{y} - \xx)$.
\end{enumerate}

\subsection{Solution}
\begin{enumerate}[label = \alph*)]
    \item 
        Recall that the definition of a partial derivative. Suppose that $f : D \mapsto \reals$ where $D \subseteq \reals^n$. Then if $\xx$ is an interior point of $D$, the partial derivative of $f$ with respect to $x_i$ at $\xx$ is defined as 
    \begin{equation}
    \frac{\partial f(x)}{\partial x_i} = \lim_{h \to 0}\frac{f(x_1, ..., x_i + h, ..., x_n) - f(x_1, ..., x_i, ..., x_n)}{h} \label{partial_definition}
    \end{equation}
    We are asked to show that the partial derivative of the convex combination of two bundles $(1 - t) \xx + t \mathbf{y}$ with regards to its modulus $t$, evaluated at the point $t = 0$ is non-negative. Formally, we write
    \begin{equation}
    \frac{\partial}{\partial t} u((1 - t) \xx + t \mathbf{y}) \Bigr\rvert_{t = 0} \geq 0 \label{2b_toprove}
    \end{equation}
    
    Generally for any $t \in [0,1]$ our object of interest can be expressed as follows, using the definition of the partial derivative in (\ref{partial_definition}),
    \begin{equation}
    \frac{\partial}{\partial t} u((1 - t) \xx + t \mathbf{y}) = \lim_{h \to 0} \frac{ u((1 - (t + h)) \xx + (t + h) \mathbf{y}) - u((1 - t) \xx + t \mathbf{y})}{h}
    \end{equation}
    Hence, if we evaluate the partial derivative at $t = 0$, the expression becomes
    \begin{align}
    \frac{\partial}{\partial t} u((1 - t) \xx + t \mathbf{y})  \Bigr\rvert_{t = 0} = \lim_{h \to 0} \frac{ u((1 - h) \xx + h\mathbf{y}) - u( \xx )}{h} \label{directional}
    \end{align}
    Now note that we can write $(1 - t) \xx + t \mathbf{y}$ as $\xx + h (\mathbf{y} - \xx)$, so the right-hand side of (\ref{directional}) is the \textit{directional derivative} of $u$ at $\xx$ in the direction of $\mathbf{y} - \xx$. Since $u$ is differentiable at $\xx$ by assumption, the directional derivative exists in all directions and this particular limit is well-defined.
    The quasiconcavity of $u$ and the assumption that $u(\mathbf{y}) \geq u(\xx)$ imply that for all $h \in [0,1]$ 
    \begin{align*}
        u((1 - h) \xx + h \mathbf{y}) - u( \xx ) &\geq \min[u(\mathbf{y}),u( \xx )] - u( \xx ) \\
        &\geq u( \xx ) - u( \xx ) \\
        &\geq 0
    \end{align*}
    This last inequality and equation (\ref{directional}) then imply (\ref{2b_toprove}).
    
    \item 
    From page 276 in Pemberton \& Rau (2016), or page 555 in Jehle \& Reny (2011), we know that the following holds
    \begin{align*}
    \frac{d u((1 - t) \xx + t \mathbf{y})}{d t} &= 
    \frac{\partial u((1 - t) \xx + t \mathbf{y})}{\partial x_1} (y_1 - x_1) \\
    &+ \frac{\partial u((1 - t) \xx + t \mathbf{y})}{\partial x_2} +  (y_2 - x_2) \\
    &... + \frac{\partial u((1 - t) \xx + t \mathbf{y})}{\partial x_n} +  (y_n - x_n) \\
    &= \sum^{n}_{i = 1} \frac{\partial u((1 - t) \xx + t \mathbf{y})}{\partial x_i} (y_i - x_i) 
    \end{align*}
    Evaluating this expression at $t = 0$, we get that
    \begin{align*}
    \frac{d u((1 - t) \xx + t \mathbf{y})}{d t} \Bigr\rvert_{t = 0} 
    &= \sum^{n}_{i = 1} \frac{\partial u(\xx)}{\partial x_i} (y_i - x_i) \\
    &= \nabla u(\xx) \cdot (\mathbf{y} - \xx)
    \end{align*}
    where $\nabla u(\xx)$ is the gradient of $u$ at $\xx$, i.e. a column vector of length $n$ containing all the partial derivatives of $u$ at $\xx$, $\nabla u(\xx) = \{\partial u(\xx) / \partial x_1, ..., \partial u(\xx) / \partial x_n\}$.\footnote{This \enquote{proof} is a bit heuristic, admittedly.}
\end{enumerate}


\end{document}