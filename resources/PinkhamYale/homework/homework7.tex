\documentclass[11pt, a4paper]{article}
%\usepackage{geometry}
\usepackage[inner=1.5cm,outer=1.5cm,top=2.5cm,bottom=2.5cm]{geometry}
\pagestyle{empty}
\usepackage{graphicx}
\usepackage{amsmath, amssymb}
\usepackage[usenames,dvipsnames]{color}
\definecolor{darkblue}{rgb}{0,0,.6}
\definecolor{darkred}{rgb}{.7,0,0}
\definecolor{darkgreen}{rgb}{0,.6,0}
\definecolor{red}{rgb}{.98,0,0}
\usepackage[colorlinks,pagebackref,pdfusetitle,urlcolor=darkblue,citecolor=darkblue,linkcolor=darkred,bookmarksnumbered,plainpages=false]{hyperref}
\renewcommand{\thefootnote}{\fnsymbol{footnote}}



\thispagestyle{plain}

%%%%%%%%%%%% LISTING %%%
\usepackage{listings}
\usepackage{caption}
\DeclareCaptionFont{white}{\color{white}}
\DeclareCaptionFormat{listing}{\colorbox{gray}{\parbox{\textwidth}{#1#2#3}}}
\captionsetup[lstlisting]{format=listing,labelfont=white,textfont=white}
\usepackage{verbatim} % used to display code
\usepackage{fancyvrb}
\usepackage{acronym}
\usepackage{amsthm}
\VerbatimFootnotes % Required, otherwise verbatim does not work in footnotes!



\definecolor{OliveGreen}{cmyk}{0.64,0,0.95,0.40}
\definecolor{CadetBlue}{cmyk}{0.62,0.57,0.23,0}
\definecolor{lightlightgray}{gray}{0.93}



\lstset{
%language=bash,                          % Code langugage
basicstyle=\ttfamily,                   % Code font, Examples: \footnotesize, \ttfamily
keywordstyle=\color{OliveGreen},        % Keywords font ('*' = uppercase)
commentstyle=\color{gray},              % Comments font
numbers=left,                           % Line nums position
numberstyle=\tiny,                      % Line-numbers fonts
stepnumber=1,                           % Step between two line-numbers
numbersep=5pt,                          % How far are line-numbers from code
backgroundcolor=\color{lightlightgray}, % Choose background color
frame=none,                             % A frame around the code
tabsize=2,                              % Default tab size
captionpos=t,                           % Caption-position = bottom
breaklines=true,                        % Automatic line breaking?
breakatwhitespace=false,                % Automatic breaks only at whitespace?
showspaces=false,                       % Dont make spaces visible
showtabs=false,                         % Dont make tabls visible
columns=flexible,                       % Column format
morekeywords={__global__, __device__},  % CUDA specific keywords
}

%%%%%%%%%%%%%%%%%%%%%%%%%%%%%%%%%%%%
\begin{document}
\begin{center}
  {\Large \textsc{Problem Set 7: Instrumental Variables}}\\
  MGMT 737
\end{center}

\begin{enumerate}
\item \textbf{Compliers and Weak IV}: We explore Angrist and Evan (1998), who
  study the impact of children on parents' supply. This paper has two
  instruments: 1) whether the first two children of a mother are the
  same sex \textit{Same sex}; 2) whether the second (successful)
  pregnancy is twins \textit{Twins-2}. The 1980 Data for this paper is
  available in the repository in a csv \texttt{ang\_ev\_1980\.csv}.

  I was not able to perfectly match AE's sample but it is very
  close. As a result, the point estimates are not identical, but are
  close enough that you should be able to feel confident. You may use pre-built estimation packages unless otherwise stated.

  \textit{More than 2 children} is \texttt{morekids}, \textit{Number
    of children} is \texttt{kidcount}, \textit{worked for pay} is
  \texttt{mom\_worked}, \textit{Weeks worked} is
  \texttt{mom\_weeks\_worked} and the two dummy instruments are
  \texttt{samesex} and \texttt{twins\_2}.
  \begin{enumerate}
  \item Replicate the coefficients from Table 5, Column 1, rows 1-4, using a linear regression.
  \item Replicate the coefficients from Table 5, Column 2, rows 3-4,
    using 2SLS. Convince yourself you could construct this estimate by
    hand using the result in the previous answer.
  \item Replicate the coefficients from Table 5, Column 7, rows 1-4, using a linear regression.
  \item For the endogeneous vairable ``More than 2 children'', what is
    the complier share for each of the two instruments?
  \item For the endogeneous vairable ``More than 2 children'' and each
    of the two instruments, what is the average share of the complier
    population with an education greater than high school
    (\texttt{moreths})? What about different mother race shares?
  \item Using the Same sex instrument, construct the Weak IV robust
    Anderson-Rubin confidence intervals using the algorithm outlined
    in Chernozhukov and Hansen (2007) (see the slides)
  \end{enumerate}
\item \textbf{Kitagawa test} Now we implement the Kitagawa
  (2015) test for instrument validity. We will focus on the
  \textit{worked for pay} outcome as our endogeneous variable $Y$,
  more than two children as our endogeneous variable, and our two
  instruments will be as before (same sex and twins). Implement the procedure below for both instruments.
  \begin{enumerate}
  \item Let
    $P(y,d) = n_{1}^{-1} \sum_{i : Z_{i} = 1} 1(Y_{i} = y, D_{i} = d)$,
    where $n_{1}$ is the number of observations where $Z_{i} = 1$ in
    the sample, and
    $Q(y,d) = n_{0}^{-1} \sum_{i : Z_{i} = 0} 1(Y_{i} = y, D_{i} = d)$,
    where $n_{1}$ is the number of observations where $Z_{i} = 0$ in
    the sample. $n = n_{0} + n_{1}$ Estimate $P(y,d)$ and $Q(y,d)$ and
    report them for the 4 possible pairs of $(y,d)$.
  \item Next, we want to calculate the following statistic (Note that this is simplified relative to the general Kitagawa case because $Y$ is binary):
    \begin{align*}
      T = \left(\frac{n_{0}n_{1}}{n}\right)^{1/2} \max \bigl\{&\sup_{y\in \{0,1\}} \left\{\frac{Q(y,1) - P(y,1)}{\xi \vee \sigma_{P,Q}(y,1)}\right\},\\
      &\sup_{y\in \{0,1\}} \left\{\frac{Q(y,0) - P(y,0)}{\xi \vee \sigma_{P,Q}(y,0)}\right\}\bigr\}
    \end{align*}
    where $\xi$ is a trimming constant that you pick, $\vee$ denotes the ``max'' between the two objects, and
    \begin{equation*}
      \sigma^{2}_{P,Q}(y,d) = (1-\hat{\lambda})P(y,d)(1-P(y,d)) + \hat{\lambda}Q(y,d)(1-Q(y,d)), \qquad \qquad \hat{\lambda} = n_{1}/n.
    \end{equation*}
    Estimate $\sigma$ for each of the four possible pairs of $(y,d)$,
    and then calculate the $T$ based using a $\xi = 0.1$. Report this T statistic.
  \item Now, we need to construct critical values for the statistic
    $T$. First, take the subsample of $n_{1}$ observations where
    $Z_{i}=1$. Sample a new set of $n_{1}$ observations from this
    subsample, with replacement using weights
    $H(y,d) = \hat{\lambda}P + (1-\hat{\lambda})Q$. (E.g. the weight
    of drawing each observation is a function of this combined
    probability). Use this new data to construct a new empirical
    distribution $P^{*}$. Similarly construct a new sample of $n_{0}$
    observations based on the subsample of observations where
    $Z_{0} = 0$ using $H(y,d)$. Use this to construct $Q^{*}$. Then,
    calculate the $T$ statistic above, but using $P^{*}$ and $Q^{*}$
    in place of $P$ and $Q$.
  \item Repeat this last process 500 times and construct a
    distribution of the $T^{*}$. We reject the null of instrument
    validity (exclusion and monotonicity) at the 5 percent level if
    $T$ is greater than the 95th percentile of the distribution of
    $T^{*}$. Identify what quantile $T$ sits in the distribution of
    $T^{*}$.
  \end{enumerate}
  % For the future
% \item PART III: \textbf{Weak and Many IV} In this problem, we will use simulated data. You should use built-in software to estimate 2SLS.
%   \begin{itemize}
%   \item First, consider just-identified weak instruments. Simulate data where $\epsilon_{i}$ and $\eta_{i}$ are normally distributed, 
%   \end{itemize}
\end{enumerate}
  \end{document}