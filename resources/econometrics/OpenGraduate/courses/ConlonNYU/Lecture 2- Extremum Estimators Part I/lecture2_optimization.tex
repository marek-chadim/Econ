% xcolor and define colors -------------------------
\usepackage{xcolor}

% https://www.viget.com/articles/color-contrast/
\definecolor{purple}{HTML}{5601A4}
\definecolor{navy}{HTML}{0D3D56}
\definecolor{ruby}{HTML}{9a2515}
\definecolor{alice}{HTML}{107895}
\definecolor{daisy}{HTML}{EBC944}
\definecolor{coral}{HTML}{F26D21}
\definecolor{kelly}{HTML}{829356}
\definecolor{cranberry}{HTML}{E64173}
\definecolor{jet}{HTML}{131516}
\definecolor{asher}{HTML}{555F61}
\definecolor{slate}{HTML}{314F4F}

% Mixtape Sessions
\definecolor{picton-blue}{HTML}{00b7ff}
\definecolor{violet-red}{HTML}{ff3881}
\definecolor{sun}{HTML}{ffaf18}
\definecolor{electric-violet}{HTML}{871EFF}

\newcommand\pictonBlue[1]{{\color{picton-blue}#1}}
\newcommand\sun[1]{{\color{sun}#1}}
\newcommand\electricViolet[1]{{\color{electric-violet}#1}}
\newcommand\violetRed[1]{{\color{violet-red}#1}}

\newcommand\bgPictonBlue[1]{{\colorbox{picton-blue!20!white}{#1}}}
\newcommand\bgSun[1]{{\colorbox{sun!20!white}{#1}}}
\newcommand\bgElectricViolet[1]{{\colorbox{electric-violet!20!white}{#1}}}
\newcommand\bgVioletRed[1]{{\colorbox{violet-red!20!white}{#1}}}

\def\code#1{\texttt{#1}}

% Main theme colors
\definecolor{accent}{HTML}{00b7ff}
\definecolor{accent2}{HTML}{871EFF}
\definecolor{gray100}{HTML}{f3f4f6}
\definecolor{gray800}{HTML}{1F292D}


% Beamer Options -------------------------------------

% Background
\setbeamercolor{background canvas}{bg = white}

% Change text margins
\setbeamersize{text margin left = 15pt, text margin right = 15pt} 

% \alert
\setbeamercolor{alerted text}{fg = accent2}

% Frame title
\setbeamercolor{frametitle}{bg = white, fg = jet}
\setbeamercolor{framesubtitle}{bg = white, fg = accent}
\setbeamerfont{framesubtitle}{size = \small, shape = \itshape}

% Block
\setbeamercolor{block title}{fg = white, bg = accent2}
\setbeamercolor{block body}{fg = gray800, bg = gray100}

% Title page
\setbeamercolor{title}{fg = gray800}
\setbeamercolor{subtitle}{fg = accent}

%% Custom \maketitle and \titlepage
\setbeamertemplate{title page}
{
    %\begin{centering}
        \vspace{20mm}
        {\Large \usebeamerfont{title}\usebeamercolor[fg]{title}\inserttitle}\\
        {\large \itshape \usebeamerfont{subtitle}\usebeamercolor[fg]{subtitle}\insertsubtitle}\\ \vspace{10mm}
        {\insertauthor}\\
        {\color{asher}\small{\insertdate}}\\
    %\end{centering}
}

% Table of Contents
\setbeamercolor{section in toc}{fg = accent!70!jet}
\setbeamercolor{subsection in toc}{fg = jet}

% Button 
\setbeamercolor{button}{bg = accent}

% Remove navigation symbols
\setbeamertemplate{navigation symbols}{}

% Table and Figure captions
\setbeamercolor{caption}{fg=jet!70!white}
\setbeamercolor{caption name}{fg=jet}
\setbeamerfont{caption name}{shape = \itshape}

% Bullet points

%% Fix spacing between items
\let\olditemize=\itemize 
\let\endolditemize=\enditemize 
\renewenvironment{itemize}{\vspace{0.25em}\olditemize \itemsep0.25em}{\endolditemize}

%% Fix left-margins
\settowidth{\leftmargini}{\usebeamertemplate{itemize item}}
\addtolength{\leftmargini}{\labelsep}

%% enumerate item color
\setbeamercolor{enumerate item}{fg = accent}
\setbeamerfont{enumerate item}{size = \small}
\setbeamertemplate{enumerate item}{\insertenumlabel.}

%% itemize
\setbeamercolor{itemize item}{fg = accent!70!white}
\setbeamerfont{itemize item}{size = \small}
\setbeamertemplate{itemize item}[circle]

%% right arrow for subitems
\setbeamercolor{itemize subitem}{fg = accent!60!white}
\setbeamerfont{itemize subitem}{size = \small}
\setbeamertemplate{itemize subitem}{$\rightarrow$}

\setbeamertemplate{itemize subsubitem}[square]
\setbeamercolor{itemize subsubitem}{fg = jet}
\setbeamerfont{itemize subsubitem}{size = \small}








% Links ----------------------------------------------

\usepackage{hyperref}
\hypersetup{
  colorlinks = true,
  linkcolor = accent2,
  filecolor = accent2,
  urlcolor = accent2,
  citecolor = accent2,
}


% Line spacing --------------------------------------
\usepackage{setspace}
\setstretch{1.35}


% \begin{columns} -----------------------------------
\usepackage{multicol}


% Fonts ---------------------------------------------
% Beamer Option to use custom fonts
\usefonttheme{professionalfonts}

% \usepackage[utopia, smallerops, varg]{newtxmath}
% \usepackage{utopia}
\usepackage[sfdefault,light]{roboto}

% Small adjustments to text kerning
\usepackage{microtype}



% Remove annoying over-full box warnings -----------
\vfuzz2pt 
\hfuzz2pt


% Table of Contents with Sections
\setbeamerfont{myTOC}{series=\bfseries, size=\Large}
\AtBeginSection[]{
        \frame{
            \frametitle{Roadmap}
            \tableofcontents[current]   
        }
    }


% Tables -------------------------------------------
% Tables too big
% \begin{adjustbox}{width = 1.2\textwidth, center}
\usepackage{adjustbox}
\usepackage{array}
\usepackage{threeparttable, booktabs, adjustbox}
    
% Fix \input with tables
% \input fails when \\ is at end of external .tex file
\makeatletter
\let\input\@@input
\makeatother

% Tables too narrow
% \begin{tabularx}{\linewidth}{cols}
% col-types: X - center, L - left, R -right
% Relative scale: >{\hsize=.8\hsize}X/L/R
\usepackage{tabularx}
\newcolumntype{L}{>{\raggedright\arraybackslash}X}
\newcolumntype{R}{>{\raggedleft\arraybackslash}X}
\newcolumntype{C}{>{\centering\arraybackslash}X}

% Figures

% \imageframe{img_name} -----------------------------
% from https://github.com/mattjetwell/cousteau
\newcommand{\imageframe}[1]{%
    \begin{frame}[plain]
        \begin{tikzpicture}[remember picture, overlay]
            \node[at = (current page.center), xshift = 0cm] (cover) {%
                \includegraphics[keepaspectratio, width=\paperwidth, height=\paperheight]{#1}
            };
        \end{tikzpicture}
    \end{frame}%
}

% subfigures
\usepackage{subfigure}


% Highlight slide -----------------------------------
% \begin{transitionframe} Text \end{transitionframe}
% from paulgp's beamer tips
\newenvironment{transitionframe}{
    \setbeamercolor{background canvas}{bg=accent!40!black}
    \begin{frame}\color{accent!10!white}\LARGE\centering
}{
    \end{frame}
}


% Table Highlighting --------------------------------
% Create top-left and bottom-right markets in tabular cells with a unique matching id and these commands will outline those cells
\usepackage[beamer,customcolors]{hf-tikz}
\usetikzlibrary{calc}
\usetikzlibrary{fit,shapes.misc}

% To set the hypothesis highlighting boxes red.
\newcommand\marktopleft[1]{%
    \tikz[overlay,remember picture] 
        \node (marker-#1-a) at (0,1.5ex) {};%
}
\newcommand\markbottomright[1]{%
    \tikz[overlay,remember picture] 
        \node (marker-#1-b) at (0,0) {};%
    \tikz[accent!80!jet, ultra thick, overlay, remember picture, inner sep=4pt]
        \node[draw, rectangle, fit=(marker-#1-a.center) (marker-#1-b.center)] {};%
}


% DAGS ----------------------------------------------
\usepackage{tikz}
\usetikzlibrary{shapes,decorations,arrows,calc,arrows.meta,fit,positioning}
% Tikz settings optimized for causal graphs.
\tikzset{
    -Latex,auto,node distance =1 cm and 1 cm,semithick,
    state/.style ={ellipse, draw, minimum width = 0.7 cm},
    point/.style = {circle, draw, inner sep=0.04cm,fill,node contents={}},
    bidirected/.style={Latex-Latex,dashed},
    el/.style = {inner sep=2pt, align=left, sloped}
}


% Beamer tricks -------------------------------------
% Make \pause work in align environments
\makeatletter
\renewrobustcmd{\beamer@@pause}[1][]{%
  \unless\ifmeasuring@%
  \ifblank{#1}%
    {\stepcounter{beamerpauses}}%
    {\setcounter{beamerpauses}{#1}}%
  \onslide<\value{beamerpauses}->\relax%
  \fi%
}
\makeatother


% \usepackage{slashbox}
\title{Lecture 2: Maximum Likelihood and Friends}
\author{Chris Conlon }
\institute{NYU Stern }


\date{\today}

\begin{document}
\maketitle


\section*{Computing Maximum Likelihood Estimators}

\begin{frame}{Newton's Method for Root Finding}
Consider the Taylor series for $f(x)$ approximated around $f(x_0)$:
\begin{align*}
f(x) \approx f(x_0) + f'(x_0) \cdot (x-x_0) + f''(x_0) \cdot (x-x_0)^2 + o_p(3)
\end{align*}
Suppose we wanted to find a \alert{root} of the equation where $f(x^{*})=0$ and solve for $x$:
\begin{align*}
0 &= f(x_0) + f'(x_0) \cdot (x-x_0) \\
x_1 &= x_0-\frac{f(x_0)}{f'(x_0)} 
\end{align*}
This gives us an \alert{iterative} scheme to find $x^{*}$:
\begin{enumerate}
\item Start with some $x_k$. Calculate $f(x_k),f'(x_k)$
\item Update using $x_{k+1} = x_k - \frac{f(x_k)}{f'(x_k)} $
\item Stop when $|x_{k+1}-x_{k}| < \epsilon_{tol}$.
\end{enumerate}
\end{frame}

\begin{frame}{Newton-Raphson for Minimization}
We can re-write \alert{optimization} as \alert{root finding};
\begin{itemize}
\item We want to know $\hat{\theta} = \arg \max_{\theta} \ell(\theta)$.
\item Construct the FOCs $\frac{\partial \ell}{\partial \theta}=0 \rightarrow$  and find the zeros.
\item How? using Newton's method! Set $f(\theta) = \frac{\partial \ell}{\partial \theta}$
\end{itemize}
\begin{align*}
\theta_{k+1} &= \theta_k -  \left[ \frac{\partial^2 \ell}{\partial \theta^2}(\theta_k) \right]^{-1} \cdot \frac{\partial \ell}{\partial \theta}(\theta_k)
\end{align*}
The SOC is that $ \frac{\partial^2 \ell}{\partial \theta^2} >0$. Ideally at all $\theta_k$.\\
This is all for a \alert{single variable} but the \alert{multivariate} version is basically the same.
\end{frame}


\begin{frame}{Newton's Method: Multivariate}
Start with the objective $Q(\theta) = - \ell(\theta)$:
\begin{itemize}
\item Approximate $Q(\theta)$ around some initial guess $\theta_0$ with a quadratic function
\item Minimize the quadratic function (because that is easy) call that $\theta_1$
\item Update the approximation and repeat.
\begin{align*}
\theta_{k+1} = \theta_k - \left[ \frac{\partial^2 Q}{\partial \theta \partial \theta'} \right]^{-1}\frac{\partial Q}{\partial \theta}(\theta_k)
\end{align*}
\item The equivalent SOC is that the {Hessian Matrix} is \alert{positive semi-definite}  (ideally at all $\theta$).
\item In that case the problem is \alert{globally convex} and has a \alert{unique maximum} that is easy to find.
\end{itemize}
\end{frame}


\begin{frame}{Newton's Method}
We can generalize to Quasi-Newton methods:
\begin{align*}
\theta_{k+1} = \theta_k -  \lambda_k \underbrace{\left[ \frac{\partial^2 Q}{\partial \theta \partial \theta'} \right]^{-1}}_{A_k} \frac{\partial Q}{\partial \theta}(\theta_k)
\end{align*}
Two Choices:
\begin{itemize}
\item Step length $\lambda_k$
\item Step direction $d_k=A_k \frac{\partial Q}{\partial \theta}(\theta_k)$
\item Often rescale the direction to be unit length $\frac{d_k}{\norm{d_k}}$.
\item If we use $A_k$ as the true Hessian and $\lambda_k=1$ this is a \alert{full Newton step}.
\end{itemize}
\end{frame}

\begin{frame}{Newton's Method: Alternatives}
Choices for $A_k$
\begin{itemize}
\item $A_k= I_{k}$ (Identity) is known as \alert{gradient descent} or \alert{steepest descent}
\item BHHH. Specific to MLE. Exploits the \alert{Fisher Information}.
\begin{align*}
A _ { k } 
&= \left[ \frac { 1 } { N } \sum _ { i = 1 } ^ { N } \frac { \partial \ln f } { \partial \theta } \left( \theta _ { k } \right) \frac { \partial \ln f } { \partial \theta ^ { \prime } } \left( \theta _ { k } \right) \right] ^ { - 1 }\\
&=- \mathbb { E } \left[ \frac { \partial ^ { 2 } \ln f } { \partial \theta \partial \theta ^ { \prime } } \left( Z , \theta ^ { * } \right) \right] 
= \mathbb { E } \left[ \frac { \partial \ln f } { \partial \theta } \left( Z , \theta ^ { * } \right) \frac { \partial \ln f } { \partial \theta ^ { \prime } } \left( Z , \theta ^ { * } \right) \right]
\end{align*}
\item Alternatives \alert{SR1} and \alert{DFP} rely on an initial estimate of the Hessian matrix and then approximate an update to $A_k$.
\item Usually updating the Hessian is the costly step.
\item Non invertible Hessians are bad news.
\end{itemize}
\end{frame}

\section{EM Algorithm and Mixtures}

\begin{frame}
\frametitle{Estimating Finite Mixtures}
\begin{itemize}
\item In practice estimating finite mixture models can be tricky.
\item A simple example is the mixture of normals (incomplete data likelihood)
\begin{eqnarray*}
f(x_1,\ldots,x_n | \theta) = \prod_{i=1}^N \sum_{k=1}^K \pi_k f(x_i | \mu_k, \sigma_k)
\end{eqnarray*}
\item We need to find both mixture weights $\pi_k = Pr(z_k)$ and the components $(\mu_k,\sigma_k)$ the weights define a valid probabiltiy measure $\sum_k \pi_k = 1$.
\item Easy problem is \alert{label switching}. Usually it helps to order the components by say decreasing $\pi_1 > \pi_2 > \ldots$ or  $\mu_1 > \mu_2 > \ldots$ 
\item The real problem is that which component you belong to is unobserved. We can add an extra indicator variable $z_{ik} \in \{0,1\}$.
\item We don't care about $z_{ik}$ per-se so they are \alert{nuisance parameters}.
\end{itemize}
\end{frame}

\begin{frame}
\frametitle{Estimating Finite Mixtures}
\begin{itemize}
\item We can write the complete data log-likelihood (as if we observed $z_{ik}$):
\begin{eqnarray*}
\ell(x_1,\ldots,x_n | \theta) = \sum_{i=1}^N  \log \left( \sum_{k=1}^K I[z_i = k]  \pi_k f(x_i , \mu_k, \sigma_k) \right)
\end{eqnarray*}
\item We can instead maximized the expected log-likelihood where we take the expectation $E_{z|\theta}$
\begin{eqnarray*}
\alpha_{ik}(\theta) = Pr(z_{ik} =1 | x_i,\theta) = \frac{f_k(x_i,z_k,\mu_k,\sigma_k) \pi_k }{\sum_{m=1}^K f_m(x_i,z_m,\mu_m,\sigma_m) \pi_m}
\end{eqnarray*}
\item Now we have a probability $\hat{\alpha}_{ik}$ that gives us the probability that $i$ came from component $k$. We also compute $\hat{\pi}_k = \frac{1}{N} \sum_{i=1}^N \alpha_{ik}$
\end{itemize}
\end{frame}

\begin{frame}
\frametitle{EM Algorithm}
\begin{itemize}
\item Treat the $\hat{\alpha}_k(\theta^{(q)})$ as data and maximize to find $\mu_k,\sigma_k$ for each $k$
\begin{eqnarray*}
\hat{\theta}^{(q+1)} = \arg \max_{\theta}  \sum_{i=1}^N  \log \left( \sum_{k=1}^K \hat{\alpha}_k(\theta^{(q)}) f(x_i | z_{ik}, \theta ) \right)
\end{eqnarray*}
\item We iterate between updating $\hat{\alpha}_k(\theta^{(q)})$ (E-step) and $\hat{\theta}^{(q+1)}$ (M-step)
\item For the mixture of normals we can compute the M-step very easily:
\begin{eqnarray*}
\mu_k^{(q+1)} &=& \frac{1}{N} \sum_{i=1}^N \hat{\alpha}_k(\theta^{(q)}) x_{i}\\
\sigma_k^{(q+1)} &=& \frac{1}{N} \sum_{i=1}^N \hat{\alpha}_k(\theta^{(q)}) (x_{i} - \overline{x})^2 \\
\end{eqnarray*}
\end{itemize}
\end{frame}

\begin{frame}
\frametitle{EM Algorithm}
\begin{itemize}
\item EM algorithm has the advantage that it avoids complicated integrals in computing the expected log-likelihood over the missing data.
\item For a large set of families it is proven to converge to the MLE
\item That convergence is \alert{monotonic} and \alert{linear}. (Newton's method is quadratic)
\item This means it can be slow, but sometimes $\nabla_{\theta} f (\cdot)$ is really complicated.
\end{itemize}
\end{frame}

\section*{Thanks!}

\end{document}