% xcolor and define colors -------------------------
\usepackage{xcolor}

% https://www.viget.com/articles/color-contrast/
\definecolor{purple}{HTML}{5601A4}
\definecolor{navy}{HTML}{0D3D56}
\definecolor{ruby}{HTML}{9a2515}
\definecolor{alice}{HTML}{107895}
\definecolor{daisy}{HTML}{EBC944}
\definecolor{coral}{HTML}{F26D21}
\definecolor{kelly}{HTML}{829356}
\definecolor{cranberry}{HTML}{E64173}
\definecolor{jet}{HTML}{131516}
\definecolor{asher}{HTML}{555F61}
\definecolor{slate}{HTML}{314F4F}

% Mixtape Sessions
\definecolor{picton-blue}{HTML}{00b7ff}
\definecolor{violet-red}{HTML}{ff3881}
\definecolor{sun}{HTML}{ffaf18}
\definecolor{electric-violet}{HTML}{871EFF}

\newcommand\pictonBlue[1]{{\color{picton-blue}#1}}
\newcommand\sun[1]{{\color{sun}#1}}
\newcommand\electricViolet[1]{{\color{electric-violet}#1}}
\newcommand\violetRed[1]{{\color{violet-red}#1}}

\newcommand\bgPictonBlue[1]{{\colorbox{picton-blue!20!white}{#1}}}
\newcommand\bgSun[1]{{\colorbox{sun!20!white}{#1}}}
\newcommand\bgElectricViolet[1]{{\colorbox{electric-violet!20!white}{#1}}}
\newcommand\bgVioletRed[1]{{\colorbox{violet-red!20!white}{#1}}}

\def\code#1{\texttt{#1}}

% Main theme colors
\definecolor{accent}{HTML}{00b7ff}
\definecolor{accent2}{HTML}{871EFF}
\definecolor{gray100}{HTML}{f3f4f6}
\definecolor{gray800}{HTML}{1F292D}


% Beamer Options -------------------------------------

% Background
\setbeamercolor{background canvas}{bg = white}

% Change text margins
\setbeamersize{text margin left = 15pt, text margin right = 15pt} 

% \alert
\setbeamercolor{alerted text}{fg = accent2}

% Frame title
\setbeamercolor{frametitle}{bg = white, fg = jet}
\setbeamercolor{framesubtitle}{bg = white, fg = accent}
\setbeamerfont{framesubtitle}{size = \small, shape = \itshape}

% Block
\setbeamercolor{block title}{fg = white, bg = accent2}
\setbeamercolor{block body}{fg = gray800, bg = gray100}

% Title page
\setbeamercolor{title}{fg = gray800}
\setbeamercolor{subtitle}{fg = accent}

%% Custom \maketitle and \titlepage
\setbeamertemplate{title page}
{
    %\begin{centering}
        \vspace{20mm}
        {\Large \usebeamerfont{title}\usebeamercolor[fg]{title}\inserttitle}\\
        {\large \itshape \usebeamerfont{subtitle}\usebeamercolor[fg]{subtitle}\insertsubtitle}\\ \vspace{10mm}
        {\insertauthor}\\
        {\color{asher}\small{\insertdate}}\\
    %\end{centering}
}

% Table of Contents
\setbeamercolor{section in toc}{fg = accent!70!jet}
\setbeamercolor{subsection in toc}{fg = jet}

% Button 
\setbeamercolor{button}{bg = accent}

% Remove navigation symbols
\setbeamertemplate{navigation symbols}{}

% Table and Figure captions
\setbeamercolor{caption}{fg=jet!70!white}
\setbeamercolor{caption name}{fg=jet}
\setbeamerfont{caption name}{shape = \itshape}

% Bullet points

%% Fix spacing between items
\let\olditemize=\itemize 
\let\endolditemize=\enditemize 
\renewenvironment{itemize}{\vspace{0.25em}\olditemize \itemsep0.25em}{\endolditemize}

%% Fix left-margins
\settowidth{\leftmargini}{\usebeamertemplate{itemize item}}
\addtolength{\leftmargini}{\labelsep}

%% enumerate item color
\setbeamercolor{enumerate item}{fg = accent}
\setbeamerfont{enumerate item}{size = \small}
\setbeamertemplate{enumerate item}{\insertenumlabel.}

%% itemize
\setbeamercolor{itemize item}{fg = accent!70!white}
\setbeamerfont{itemize item}{size = \small}
\setbeamertemplate{itemize item}[circle]

%% right arrow for subitems
\setbeamercolor{itemize subitem}{fg = accent!60!white}
\setbeamerfont{itemize subitem}{size = \small}
\setbeamertemplate{itemize subitem}{$\rightarrow$}

\setbeamertemplate{itemize subsubitem}[square]
\setbeamercolor{itemize subsubitem}{fg = jet}
\setbeamerfont{itemize subsubitem}{size = \small}








% Links ----------------------------------------------

\usepackage{hyperref}
\hypersetup{
  colorlinks = true,
  linkcolor = accent2,
  filecolor = accent2,
  urlcolor = accent2,
  citecolor = accent2,
}


% Line spacing --------------------------------------
\usepackage{setspace}
\setstretch{1.35}


% \begin{columns} -----------------------------------
\usepackage{multicol}


% Fonts ---------------------------------------------
% Beamer Option to use custom fonts
\usefonttheme{professionalfonts}

% \usepackage[utopia, smallerops, varg]{newtxmath}
% \usepackage{utopia}
\usepackage[sfdefault,light]{roboto}

% Small adjustments to text kerning
\usepackage{microtype}



% Remove annoying over-full box warnings -----------
\vfuzz2pt 
\hfuzz2pt


% Table of Contents with Sections
\setbeamerfont{myTOC}{series=\bfseries, size=\Large}
\AtBeginSection[]{
        \frame{
            \frametitle{Roadmap}
            \tableofcontents[current]   
        }
    }


% Tables -------------------------------------------
% Tables too big
% \begin{adjustbox}{width = 1.2\textwidth, center}
\usepackage{adjustbox}
\usepackage{array}
\usepackage{threeparttable, booktabs, adjustbox}
    
% Fix \input with tables
% \input fails when \\ is at end of external .tex file
\makeatletter
\let\input\@@input
\makeatother

% Tables too narrow
% \begin{tabularx}{\linewidth}{cols}
% col-types: X - center, L - left, R -right
% Relative scale: >{\hsize=.8\hsize}X/L/R
\usepackage{tabularx}
\newcolumntype{L}{>{\raggedright\arraybackslash}X}
\newcolumntype{R}{>{\raggedleft\arraybackslash}X}
\newcolumntype{C}{>{\centering\arraybackslash}X}

% Figures

% \imageframe{img_name} -----------------------------
% from https://github.com/mattjetwell/cousteau
\newcommand{\imageframe}[1]{%
    \begin{frame}[plain]
        \begin{tikzpicture}[remember picture, overlay]
            \node[at = (current page.center), xshift = 0cm] (cover) {%
                \includegraphics[keepaspectratio, width=\paperwidth, height=\paperheight]{#1}
            };
        \end{tikzpicture}
    \end{frame}%
}

% subfigures
\usepackage{subfigure}


% Highlight slide -----------------------------------
% \begin{transitionframe} Text \end{transitionframe}
% from paulgp's beamer tips
\newenvironment{transitionframe}{
    \setbeamercolor{background canvas}{bg=accent!40!black}
    \begin{frame}\color{accent!10!white}\LARGE\centering
}{
    \end{frame}
}


% Table Highlighting --------------------------------
% Create top-left and bottom-right markets in tabular cells with a unique matching id and these commands will outline those cells
\usepackage[beamer,customcolors]{hf-tikz}
\usetikzlibrary{calc}
\usetikzlibrary{fit,shapes.misc}

% To set the hypothesis highlighting boxes red.
\newcommand\marktopleft[1]{%
    \tikz[overlay,remember picture] 
        \node (marker-#1-a) at (0,1.5ex) {};%
}
\newcommand\markbottomright[1]{%
    \tikz[overlay,remember picture] 
        \node (marker-#1-b) at (0,0) {};%
    \tikz[accent!80!jet, ultra thick, overlay, remember picture, inner sep=4pt]
        \node[draw, rectangle, fit=(marker-#1-a.center) (marker-#1-b.center)] {};%
}


% DAGS ----------------------------------------------
\usepackage{tikz}
\usetikzlibrary{shapes,decorations,arrows,calc,arrows.meta,fit,positioning}
% Tikz settings optimized for causal graphs.
\tikzset{
    -Latex,auto,node distance =1 cm and 1 cm,semithick,
    state/.style ={ellipse, draw, minimum width = 0.7 cm},
    point/.style = {circle, draw, inner sep=0.04cm,fill,node contents={}},
    bidirected/.style={Latex-Latex,dashed},
    el/.style = {inner sep=2pt, align=left, sloped}
}


% Beamer tricks -------------------------------------
% Make \pause work in align environments
\makeatletter
\renewrobustcmd{\beamer@@pause}[1][]{%
  \unless\ifmeasuring@%
  \ifblank{#1}%
    {\stepcounter{beamerpauses}}%
    {\setcounter{beamerpauses}{#1}}%
  \onslide<\value{beamerpauses}->\relax%
  \fi%
}
\makeatother


% \usepackage{slashbox}
\title{Lecture 3: Generalized Method of Moments}
\author{Chris Conlon }
\institute{NYU Stern }


\date{\today}

\begin{document}
\maketitle

\begin{frame}{GMM: Intro}
In the most basic setup we begin with some data $w_i$ where $i=1,\ldots,N$. Our economic model provides the following restriction on our data:
\begin{align*}
\mathbb{E}[g(w_i, \theta_0) ] =0
\end{align*}
\begin{itemize}
\item At the true parameter value $\theta_0\in \mathbb{R}^k$ our moment conditions $g(w_i,\theta)$ are on average equal to zero. 
\item What does ``on average'' mean?  In theory, $g(w_i,\theta_0)$ is a random variable and we are making a statement about its first moment.  This is what we mean when we write $\mathbb{E}[\cdot]$.
\end{itemize}
 \end{frame}
 
 \begin{frame}{GMM: IID Normal}
 Let's estimate the parameters of an IID normal $(x_1, \ldots, x_n)$. Recall the moments of the normal:
 \begin{align*}
 \mathbb{E}[X_i] = \mu \quad \mathbb{E}[X_i^2] = \mu^2 + \sigma^2
 \end{align*}
 We could form two moments by solving the expressions above for zero:
 \begin{align*}
 g_n^1(x_1,\ldots,x_n, \mu,\sigma)  &=  \left(\frac{1}{n} \sum_{i=1}^n x_i\right) - \mu\\
 g_n^2(x_1,\ldots,x_n, \mu,\sigma)  &=  \left(\frac{1}{n} \sum_{i=1}^n x_i^2 \right) - \mu^2 - \sigma^2
% g_n^{2'}(x_1,\ldots,x_n, \mu,\sigma)  &=  \left(\frac{1}{n} \sum_{i=1}^n x_i^2 \right) - \left(\frac{1}{n} \sum_{i=1}^n x_i\right)^2- \sigma^2
 \end{align*}
 This gives us two equations and two unknowns which we can solve for $(\mu,\sigma^2)$.\\
 Of course you probably knew how to estimate the parameters of a normal...
 \end{frame}
 
 \begin{frame}{GMM: IID Normal}
 Let's estimate the parameters of an IID normal $(x_1, \ldots, x_n)$. Recall the moments of the normal:
 \begin{align*}
 \mathbb{E}[X_i] = \mu \quad \mathbb{E}[X_i^2] = \mu^2 + \sigma^2
 \end{align*}
 We could form two moments by solving the expressions above for zero:
 \begin{align*}
 g_n^1(x_1,\ldots,x_n, \mu,\sigma)  &=  \left(\frac{1}{n} \sum_{i=1}^n x_i\right) - \mu\\
 %g_n^2(x_1,\ldots,x_n, \mu,\sigma)  &=  \left(\frac{1}{n} \sum_{i=1}^n x_i^2 \right) - \mu^2 - \sigma^2\\
 g_n^{2'}(x_1,\ldots,x_n, \mu,\sigma)  &=  \left(\frac{1}{n} \sum_{i=1}^n x_i^2 \right) - \left(\frac{1}{n} \sum_{i=1}^n x_i\right)^2- \sigma^2
 \end{align*}
 This gives us two equations and two unknowns which we can solve for $(\mu,\sigma^2)$.\\
 Of course you probably knew how to estimate the parameters of a normal...
 \end{frame}
 
\begin{frame}{GMM: Sample Moments}
In practice, it is helpful to consider the sample analogue, which we abbreviate with the shorthand $g_N(\theta) \in \mathbb{R}^q$, where $g_N(\theta)$ is a $q$-dimensional vector of moment conditions.
\begin{eqnarray*}
\mathbb{E}[g(w_i, \theta )] \approx \frac{1}{N} \sum_{i=1}^N g(w_i, \theta)  \equiv g_N(\theta)
\end{eqnarray*}
\end{frame}

\begin{frame}{Other Definitions}
\begin{itemize}
\item We define the Jacobian: $D(\theta) \equiv \mathbb{E}[\frac{\partial g(w_i,\theta)}{\partial \theta}]$, which is a $q \times k$ matrix.
\item  Evaluated at the optimum, $\frac{1}{\sqrt{N}} \sum_{i=1}^N g(w_i,\theta_0) \overset{d}{\to} N(0,S)$ where $S = E[g(w_i,\theta_0) g(w_i,\theta_0)']$ is a $q \times q$ matrix.\footnote{Technical conditions to establish this are written down later.} 
\item In other words, the moment conditions which are $0$ in expectation at $\theta_0$ are normally distributed with some covariance $S$
\item Later, we will refer to a weighting matrix $W_N$ which is a $q \times q$ positive semi-definite matrix. It tells us how much to penalize the violations of one moment condition relative to another (in quadratic distance).
\end{itemize}
\end{frame}

\begin{frame}{Examples}
It is easy to see some very simple examples:
\begin{description} 
\item[OLS] Here $y_i = x_i \beta + \epsilon_i$. Exogeneity implies that $\E[x_i' \epsilon_i]=0$. We can write this in terms of just observables and parameters as $E[x_i' (y_i - x_i \beta)]=0$ so that $g(y_i,x_i, \beta) = x_i' (y_i - x_i \beta)$.
\item[IV]  Again $y_i = x_i \beta + \epsilon_i$. Now, endogeneity implies that $\E[x_i' \epsilon_i]\neq0$. However there are some instruments $z_i$ which may be partly contained in $x_i$ and partly excluded from $y_i$, so that $\E[z_i' \epsilon_i]=0$. $\E[z_i' (y_i - x_i \beta)]=0$ so that $g(y_i,x_i,z_i, \beta) = z_i' (y_i - x_i \beta)$.
\end{description}
\end{frame}

\begin{frame}{Examples (continued)}
\textbf{Maximum Likelihood}\\
 $g(w_i,\theta) =  \frac{\partial \log f(w_i,\theta)}{\partial \theta}$ where $f(w_i,\theta)$ is the density function so that $\log f(w_i,\theta)$ is the contribution of observation $i$ to the log-likelihood. Here we set the expected (average) derivative of the log-likelihood (score) function to zero.
\end{frame}


\begin{frame}{Examples (continued)}
\textbf{Euler Equations}\\
Assume we have a CRRA utility function $u(c) = \frac{c^{1-\gamma} -1}{1-\gamma}$ and an agent who maximizes the expected discounted value of their stream of consumption. This leads to an Euler Equation:
\begin{eqnarray*}
\E \left[\beta \left( \frac{c_{t+1}}{c_t} \right)^{-\gamma} R_{t+1} -1 | \Omega_t \right] =0
\end{eqnarray*}
where $\Omega_t$ is the ``Information Set'' (sigma algebra) of everything known to the agent up until time $t$ (include full histories). We can write a moment restriction of the form for any measurable $z_t \in \Omega_t$.
\begin{eqnarray*}
\E \left[z_t \left(\beta  \left( \frac{c_{t+1}}{c_t} \right)^{-\gamma} R_{t+1} -1\right) \right] =0
\end{eqnarray*}
In the original work by Hansen (1982) on GMM, this $g(c_t,c_{t+1},R_{t+1},\beta,\gamma)$ was used to estimate $(\beta,\gamma)$.
\end{frame}

\begin{frame}{GMM Estimator}
Here is the GMM estimator:
\begin{eqnarray*}
\hat{\theta} = \arg \min_{\theta}  Q_N(\theta) \quad Q_N(\theta)=g_N(\theta)' W_N  g_N(\theta)
\end{eqnarray*}

\end{frame}


\begin{frame}{Technical Conditions}
\textit{These are a set of sufficient conditions to establish consistency and asymptotic normality of the GMM estimator. These conditions are stronger than necessary, but they establish the requisite LLN and CLT.}
\begin{enumerate}
\item $\theta \in \Theta$ is compact.
\item $W_N \overset{p}{\to} W$.
\item $g_N(\theta) \overset{p}{\to} \E[g(z_i,\theta)]$ (uniformly)
\item $\E[g(z_i,\theta)]$ is continous.
\item We need that $\E[g(z_i,\theta_0)]=0$ and $W_N\, \E[g(z_i,\theta)] \neq 0$ for $\theta \neq \theta_0$ (global identification condition).
\item $g_N(\theta)$ is twice continuously differentiable about $\theta_0$.
\item $\theta_0$ is not on the boundary of $\Theta$.
\item $D(\theta_0) W D(\theta_0)'$ is invertible (non-singular).
\item $g(z_i,\theta)$ has at least two moments finite and finite derivatives at all $\theta \in \Theta$.
\end{enumerate}
\end{frame}


\begin{frame}{GMM: Asymptotics}
The first five conditions give us consistency $\hat{\theta} \overset{p}{\to} \theta_0$ as $N \rightarrow \infty$. All nine conditions give us asymptotic normality.
\begin{eqnarray*}
\sqrt{N}(\hat{\theta}-\theta)  &\overset{d}{\to}& N(0,V_{\theta})\\
V_{\theta} &=& \underbrace{(D W D')^{-1}}_{\mbox{bread}} \underbrace{(D W S W' D')}_{\mbox{filling}}\underbrace{(D W D')^{-1}}_{\mbox{bread}} 
\end{eqnarray*}
It is common to refer parts of the variance as the \textit{bread} and the \textit{filling} or \textit{meat}, together this is referred to as the \textit{sandwich} estimator of the variance.
\end{frame}

\begin{frame}{GMM: Identification}

\begin{itemize}
\item The global identification condition is difficult to understand, for the linear model we can replace it with a (local) condition on Jacobian of the moment conditions. 
\item Recall the Jacobian: $D \equiv \frac{\partial g(w_i,\theta)}{\partial \theta}$, which is a $q \times k$ matrix. 
\item We call the problem \alert{under-identified} if $rank(D) < k$, \alert{just-identified} if $rank(D) = k$ and \alert{over-identified} if $rank(D) > k$.
\begin{itemize}
\item In the under-identified case, there may be many such $\hat{\theta}$ where $g(w_i,\hat{\theta}) =0$.
\item In the just-identified case, it should be possible to find a $\hat{\theta}$ where $g_N(\hat{\theta})=0$. 
\item We are primarily interested in the over-identified case where we will generally not find $\hat{\theta}$ which satisfies the moment conditions $g_N(\hat{\theta})\neq0$.
\end{itemize}
\item Instead, we search for $\hat{\theta}$ which minimizes the violations of the moment conditions. We write this as a quadratic form for some positive definite matrix $W_N$ which is $q \times q$.
\begin{eqnarray*}
\hat{\theta} = \arg \min_{\theta}  Q_N(\theta) \quad Q_N(\theta)=g_N(\theta)' W_N  g_N(\theta)
\end{eqnarray*}
\end{itemize}
\end{frame}
\section*{Thanks!}

\end{document}