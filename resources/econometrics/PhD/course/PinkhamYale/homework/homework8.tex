\documentclass[11pt, a4paper]{article}
%\usepackage{geometry}
\usepackage[inner=1.5cm,outer=1.5cm,top=2.5cm,bottom=2.5cm]{geometry}
\pagestyle{empty}
\usepackage{graphicx}
\usepackage{amsmath, amssymb}
\usepackage[usenames,dvipsnames]{color}
\definecolor{darkblue}{rgb}{0,0,.6}
\definecolor{darkred}{rgb}{.7,0,0}
\definecolor{darkgreen}{rgb}{0,.6,0}
\definecolor{red}{rgb}{.98,0,0}
\usepackage[colorlinks,pagebackref,pdfusetitle,urlcolor=darkblue,citecolor=darkblue,linkcolor=darkred,bookmarksnumbered,plainpages=false]{hyperref}
\renewcommand{\thefootnote}{\fnsymbol{footnote}}



\thispagestyle{plain}

%%%%%%%%%%%% LISTING %%%
\usepackage{listings}
\usepackage{caption}
\DeclareCaptionFont{white}{\color{white}}
\DeclareCaptionFormat{listing}{\colorbox{gray}{\parbox{\textwidth}{#1#2#3}}}
\captionsetup[lstlisting]{format=listing,labelfont=white,textfont=white}
\usepackage{verbatim} % used to display code
\usepackage{fancyvrb}
\usepackage{acronym}
\usepackage{amsthm}
\VerbatimFootnotes % Required, otherwise verbatim does not work in footnotes!



\definecolor{OliveGreen}{cmyk}{0.64,0,0.95,0.40}
\definecolor{CadetBlue}{cmyk}{0.62,0.57,0.23,0}
\definecolor{lightlightgray}{gray}{0.93}



\lstset{
%language=bash,                          % Code langugage
basicstyle=\ttfamily,                   % Code font, Examples: \footnotesize, \ttfamily
keywordstyle=\color{OliveGreen},        % Keywords font ('*' = uppercase)
commentstyle=\color{gray},              % Comments font
numbers=left,                           % Line nums position
numberstyle=\tiny,                      % Line-numbers fonts
stepnumber=1,                           % Step between two line-numbers
numbersep=5pt,                          % How far are line-numbers from code
backgroundcolor=\color{lightlightgray}, % Choose background color
frame=none,                             % A frame around the code
tabsize=2,                              % Default tab size
captionpos=t,                           % Caption-position = bottom
breaklines=true,                        % Automatic line breaking?
breakatwhitespace=false,                % Automatic breaks only at whitespace?
showspaces=false,                       % Dont make spaces visible
showtabs=false,                         % Dont make tabls visible
columns=flexible,                       % Column format
morekeywords={__global__, __device__},  % CUDA specific keywords
}

%%%%%%%%%%%%%%%%%%%%%%%%%%%%%%%%%%%%
\begin{document}
\begin{center}
  {\Large \textsc{Problem Set 8: Regression Discontinuity}}\\
  MGMT 737
\end{center}
\begin{enumerate}
\item PART I (Comparing Estimators) We will replicate the main results
  from Table 1 of Huh and Reif (2021, AER Insights). This estimation
  will require using linear regression and also canned packages from
  two sources: Cattaneo et al.'s \texttt{rdrobust} software
  \url{https://rdpackages.github.io/rdrobust/} and Michal Kolesar's
  \texttt{rdhonest} package \url{https://github.com/kolesarm/RDHonest}.

  The code and data for this paper is available here: \url{https://github.com/reifjulian/driving}
  \begin{enumerate}
  \item Replicate the estimated point estimate and confidence
    intervals for Column 2 for the \texttt{All causes}
    outcome. (Note the Readme file on the github page -- it is quite
    easy to run this)
  \item Now, re-estimate this using a uniform kernel. How does it
    change the estimates?
  \item Re-estimate this using a local quadratic function. How does it
    change the estimates?
  \item Finally, re-estimate this using a bandwidth of 40. How does it
    change the estimate? 
  \item Describe what the ``covs'' option in the estimation approach
    is doing. How does the main estimate change if you do not include
    it?
  \item Re-estimate the main specification again, but drop
    observations where firstmonth = 1. How does this compare?
  \item Now, re-estimate this RD, dropping observations where
    firstmonth = 1, and using the RDHonest approach with an $M$
    smoothness parameter of 0.1 and the bandwidth specified by the
    Cattaneo estimator. How do the estimates and the confidence
    intervals compare?
  \item How small does the smoothness parameter need to be to match
    the confidence intervals from the original result?
    (approximately). How plausible does that seem?
  \item How does your RD Honest estimate change if you increase the
    bandwidth to 40?
  \item Finally, run the following two estimates:
    \begin{enumerate}
    \item \texttt{rdrobust} omitting observations where \texttt{firstmonth} = 1 and a uniform kernel
    \item     a linear regression of
    \begin{equation}
      \text{cod\_any} = \gamma_0 + \text{agemo\_mda}\gamma_{1} + 1(\text{agemo\_mda} > 0)\gamma_{2} + 1(\text{agemo\_mda} > 0)\times \text{agemo\_mda}\gamma_{3} + \epsilon_{it}
    \end{equation}
    using the observations that are within the bandwidth picked by the
    \texttt{rdrobust} command (hint this should be under 11, and is
    called \texttt{h}) and omitting observations where
    \texttt{firstmonth} = 1.
  \end{enumerate}
  How does your point estimate of
    $\gamma_{2}$ compare to the \texttt{rdrobust} estimate? Explain.
  \item Explain, in words, how you could get the \texttt{rdrobust}
    estimate that uses the triangular kernel using OLS (hint: it would
    involve weights)
\end{enumerate}
% \item PART II (Testing Across Estimators) Now we want to compare
%   estimated effects on Male vs. Female drivers. We will replicate
%   Columns 4 and 6 from Table 1. (You can get the data here:
%   \url{https://julianreif.com/driving/data/mortality/derived/female.dta}
%   and
%   \url{https://julianreif.com/driving/data/mortality/derived/male.dta}.
%   \begin{enumerate}
%   \item First, replicate the main point estimates and confidence
%     intervals from the outcome \texttt{All Causes} from Table 1 for
%     Male and Female drivers.
%   \item Now, we want test whether these estimates are significant
%     different from one another. 
%   \end{enumerate}
\end{enumerate}

\end{document}