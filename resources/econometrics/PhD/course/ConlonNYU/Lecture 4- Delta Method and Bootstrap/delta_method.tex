% xcolor and define colors -------------------------
\usepackage{xcolor}

% https://www.viget.com/articles/color-contrast/
\definecolor{purple}{HTML}{5601A4}
\definecolor{navy}{HTML}{0D3D56}
\definecolor{ruby}{HTML}{9a2515}
\definecolor{alice}{HTML}{107895}
\definecolor{daisy}{HTML}{EBC944}
\definecolor{coral}{HTML}{F26D21}
\definecolor{kelly}{HTML}{829356}
\definecolor{cranberry}{HTML}{E64173}
\definecolor{jet}{HTML}{131516}
\definecolor{asher}{HTML}{555F61}
\definecolor{slate}{HTML}{314F4F}

% Mixtape Sessions
\definecolor{picton-blue}{HTML}{00b7ff}
\definecolor{violet-red}{HTML}{ff3881}
\definecolor{sun}{HTML}{ffaf18}
\definecolor{electric-violet}{HTML}{871EFF}

\newcommand\pictonBlue[1]{{\color{picton-blue}#1}}
\newcommand\sun[1]{{\color{sun}#1}}
\newcommand\electricViolet[1]{{\color{electric-violet}#1}}
\newcommand\violetRed[1]{{\color{violet-red}#1}}

\newcommand\bgPictonBlue[1]{{\colorbox{picton-blue!20!white}{#1}}}
\newcommand\bgSun[1]{{\colorbox{sun!20!white}{#1}}}
\newcommand\bgElectricViolet[1]{{\colorbox{electric-violet!20!white}{#1}}}
\newcommand\bgVioletRed[1]{{\colorbox{violet-red!20!white}{#1}}}

\def\code#1{\texttt{#1}}

% Main theme colors
\definecolor{accent}{HTML}{00b7ff}
\definecolor{accent2}{HTML}{871EFF}
\definecolor{gray100}{HTML}{f3f4f6}
\definecolor{gray800}{HTML}{1F292D}


% Beamer Options -------------------------------------

% Background
\setbeamercolor{background canvas}{bg = white}

% Change text margins
\setbeamersize{text margin left = 15pt, text margin right = 15pt} 

% \alert
\setbeamercolor{alerted text}{fg = accent2}

% Frame title
\setbeamercolor{frametitle}{bg = white, fg = jet}
\setbeamercolor{framesubtitle}{bg = white, fg = accent}
\setbeamerfont{framesubtitle}{size = \small, shape = \itshape}

% Block
\setbeamercolor{block title}{fg = white, bg = accent2}
\setbeamercolor{block body}{fg = gray800, bg = gray100}

% Title page
\setbeamercolor{title}{fg = gray800}
\setbeamercolor{subtitle}{fg = accent}

%% Custom \maketitle and \titlepage
\setbeamertemplate{title page}
{
    %\begin{centering}
        \vspace{20mm}
        {\Large \usebeamerfont{title}\usebeamercolor[fg]{title}\inserttitle}\\
        {\large \itshape \usebeamerfont{subtitle}\usebeamercolor[fg]{subtitle}\insertsubtitle}\\ \vspace{10mm}
        {\insertauthor}\\
        {\color{asher}\small{\insertdate}}\\
    %\end{centering}
}

% Table of Contents
\setbeamercolor{section in toc}{fg = accent!70!jet}
\setbeamercolor{subsection in toc}{fg = jet}

% Button 
\setbeamercolor{button}{bg = accent}

% Remove navigation symbols
\setbeamertemplate{navigation symbols}{}

% Table and Figure captions
\setbeamercolor{caption}{fg=jet!70!white}
\setbeamercolor{caption name}{fg=jet}
\setbeamerfont{caption name}{shape = \itshape}

% Bullet points

%% Fix spacing between items
\let\olditemize=\itemize 
\let\endolditemize=\enditemize 
\renewenvironment{itemize}{\vspace{0.25em}\olditemize \itemsep0.25em}{\endolditemize}

%% Fix left-margins
\settowidth{\leftmargini}{\usebeamertemplate{itemize item}}
\addtolength{\leftmargini}{\labelsep}

%% enumerate item color
\setbeamercolor{enumerate item}{fg = accent}
\setbeamerfont{enumerate item}{size = \small}
\setbeamertemplate{enumerate item}{\insertenumlabel.}

%% itemize
\setbeamercolor{itemize item}{fg = accent!70!white}
\setbeamerfont{itemize item}{size = \small}
\setbeamertemplate{itemize item}[circle]

%% right arrow for subitems
\setbeamercolor{itemize subitem}{fg = accent!60!white}
\setbeamerfont{itemize subitem}{size = \small}
\setbeamertemplate{itemize subitem}{$\rightarrow$}

\setbeamertemplate{itemize subsubitem}[square]
\setbeamercolor{itemize subsubitem}{fg = jet}
\setbeamerfont{itemize subsubitem}{size = \small}








% Links ----------------------------------------------

\usepackage{hyperref}
\hypersetup{
  colorlinks = true,
  linkcolor = accent2,
  filecolor = accent2,
  urlcolor = accent2,
  citecolor = accent2,
}


% Line spacing --------------------------------------
\usepackage{setspace}
\setstretch{1.35}


% \begin{columns} -----------------------------------
\usepackage{multicol}


% Fonts ---------------------------------------------
% Beamer Option to use custom fonts
\usefonttheme{professionalfonts}

% \usepackage[utopia, smallerops, varg]{newtxmath}
% \usepackage{utopia}
\usepackage[sfdefault,light]{roboto}

% Small adjustments to text kerning
\usepackage{microtype}



% Remove annoying over-full box warnings -----------
\vfuzz2pt 
\hfuzz2pt


% Table of Contents with Sections
\setbeamerfont{myTOC}{series=\bfseries, size=\Large}
\AtBeginSection[]{
        \frame{
            \frametitle{Roadmap}
            \tableofcontents[current]   
        }
    }


% Tables -------------------------------------------
% Tables too big
% \begin{adjustbox}{width = 1.2\textwidth, center}
\usepackage{adjustbox}
\usepackage{array}
\usepackage{threeparttable, booktabs, adjustbox}
    
% Fix \input with tables
% \input fails when \\ is at end of external .tex file
\makeatletter
\let\input\@@input
\makeatother

% Tables too narrow
% \begin{tabularx}{\linewidth}{cols}
% col-types: X - center, L - left, R -right
% Relative scale: >{\hsize=.8\hsize}X/L/R
\usepackage{tabularx}
\newcolumntype{L}{>{\raggedright\arraybackslash}X}
\newcolumntype{R}{>{\raggedleft\arraybackslash}X}
\newcolumntype{C}{>{\centering\arraybackslash}X}

% Figures

% \imageframe{img_name} -----------------------------
% from https://github.com/mattjetwell/cousteau
\newcommand{\imageframe}[1]{%
    \begin{frame}[plain]
        \begin{tikzpicture}[remember picture, overlay]
            \node[at = (current page.center), xshift = 0cm] (cover) {%
                \includegraphics[keepaspectratio, width=\paperwidth, height=\paperheight]{#1}
            };
        \end{tikzpicture}
    \end{frame}%
}

% subfigures
\usepackage{subfigure}


% Highlight slide -----------------------------------
% \begin{transitionframe} Text \end{transitionframe}
% from paulgp's beamer tips
\newenvironment{transitionframe}{
    \setbeamercolor{background canvas}{bg=accent!40!black}
    \begin{frame}\color{accent!10!white}\LARGE\centering
}{
    \end{frame}
}


% Table Highlighting --------------------------------
% Create top-left and bottom-right markets in tabular cells with a unique matching id and these commands will outline those cells
\usepackage[beamer,customcolors]{hf-tikz}
\usetikzlibrary{calc}
\usetikzlibrary{fit,shapes.misc}

% To set the hypothesis highlighting boxes red.
\newcommand\marktopleft[1]{%
    \tikz[overlay,remember picture] 
        \node (marker-#1-a) at (0,1.5ex) {};%
}
\newcommand\markbottomright[1]{%
    \tikz[overlay,remember picture] 
        \node (marker-#1-b) at (0,0) {};%
    \tikz[accent!80!jet, ultra thick, overlay, remember picture, inner sep=4pt]
        \node[draw, rectangle, fit=(marker-#1-a.center) (marker-#1-b.center)] {};%
}


% DAGS ----------------------------------------------
\usepackage{tikz}
\usetikzlibrary{shapes,decorations,arrows,calc,arrows.meta,fit,positioning}
% Tikz settings optimized for causal graphs.
\tikzset{
    -Latex,auto,node distance =1 cm and 1 cm,semithick,
    state/.style ={ellipse, draw, minimum width = 0.7 cm},
    point/.style = {circle, draw, inner sep=0.04cm,fill,node contents={}},
    bidirected/.style={Latex-Latex,dashed},
    el/.style = {inner sep=2pt, align=left, sloped}
}


% Beamer tricks -------------------------------------
% Make \pause work in align environments
\makeatletter
\renewrobustcmd{\beamer@@pause}[1][]{%
  \unless\ifmeasuring@%
  \ifblank{#1}%
    {\stepcounter{beamerpauses}}%
    {\setcounter{beamerpauses}{#1}}%
  \onslide<\value{beamerpauses}->\relax%
  \fi%
}
\makeatother


\title [Bootstrap]{Delta Method}
\author{C.Conlon}
\institute{Applied Econometrics II}
\date{\today}
\setbeamerfont{equation}{size=\tiny}
\begin{document}

\begin{frame}
\titlepage
\end{frame}


\begin{frame}{Bootstrap and Delta Method}
\begin{itemize}
\item We know how to construct confidence intervals for parameter estimates:  $\hat{\theta}_k \pm 1.96 SE(\hat{\theta}_k)$
\item Often we are asked to construct standard errors or confidence intervals around model outputs that are not just parameter estimates: ie:  $h(x_i,\hat{\theta})$.
\item Sometimes we can't even write $g(x_i,\theta)$ as an explicit function of $\theta$ ie: $\Psi(h(x_i,\theta),\theta) = 0$.
\item Two options:
\begin{enumerate}
\item Delta Method
\item Bootstrap
\end{enumerate}
\end{itemize}
\end{frame}

\begin{frame}{Delta Method}
Delta method works by considering a \alert{Taylor Expansion} of $g(x_i,\theta)$.
\begin{eqnarray*}
h(z) \approx h(z_0) + h'(z_0)(z-z_0) + o(||z-z_0||)
\end{eqnarray*}
Assume that $\theta_n$ is asymptotically normally distributed so that:
\begin{eqnarray*}
\sqrt{n} (\theta_n - \theta_0) \sim N(0,\Sigma)
\end{eqnarray*}
(How do we get this: OLS? GMM? MLE?). Then we have that 
\begin{eqnarray*}
\sqrt{n} (h(\theta_n) - h(\theta_0)) \sim N(0,D(\theta)' \Sigma  D(\theta))
\end{eqnarray*}
Where $D(\theta) = \frac{\partial h(x_i, \theta)}{\partial \theta}$ is the Jacobian of $g$ with respect to theta evaluated at $\theta$.\\
We need $g$ to be continuously differentiable around the center of our expansion $\theta$.
\end{frame}


\begin{frame}{Delta Method: Examples}
Start with something simple: $g(\theta)= \overline{X}_1\cdot \overline{X}_2$ with $(X_{1i},X_{2i}) \sim IID$.
We know the CLT applies so that:
\begin{eqnarray*}
\sqrt{n}
\begin{pmatrix}
\overline{X}_1 - \mu_1\\
\overline{X}_2 - \mu_2
\end{pmatrix} &\sim  N
\begin{bmatrix}
\begin{pmatrix}
0\\
0
\end{pmatrix},
\Sigma
\end{bmatrix}
\end{eqnarray*}
The Jacobian is just $D(\theta) =  \begin{pmatrix}\frac{\partial g(\theta)}{\partial \theta_1} \\ \frac{\partial g(\theta)}{\partial \theta_2}  \end{pmatrix} =  \begin{pmatrix}s_2\\ s_1 \end{pmatrix}$\\
So,
\begin{eqnarray*}
V(Y) = D(\theta)' \Sigma D(\theta) =\begin{pmatrix} \mu_2 & \mu_1 \end{pmatrix}  \begin{pmatrix} \sigma_{11} & \sigma_{12} \\ \sigma_{21} & \sigma_{22} \end{pmatrix} \begin{pmatrix} \mu_2 \\\mu_1  
\end{pmatrix} \\
\sqrt{n} ( \overline{X}_1 \overline{X}_2 - \mu_1 \mu_2) \sim N(0,\mu_2^2 \sigma_{11}^2 + 2 \mu_1 \mu_2 \sigma_{12}  + \mu_1^2 \sigma_{22}^2)
\end{eqnarray*}
\end{frame}

\begin{frame}{Delta Method: Examples}
Think about a simple logit:
\begin{eqnarray*}
\probP(Y_i=1 | X_i ) = \frac{\exp^{\beta_0 + \beta_1 X_i}}{1+\exp^{\beta_0 + \beta_1 X_i}}  \quad \probP(Y_i=0 | X_i ) = \frac{1}{1+\exp^{\beta_0 + \beta_1 X_i}} 
\end{eqnarray*}
Remember the ``trick'' to use GLM (log-odds):
\begin{eqnarray*}
\log \probP(Y_i=1 | X_i) - \log \probP(Y_i=0 | X_i) = \beta_0 + \beta_1 X_i
\end{eqnarray*}
\begin{itemize}
\item Suppose that we have estimated $\hat{\beta_0},\hat{\beta_1}$ via GLM/MLE but we want to know the confidence interval for the probability: $P(Y_i=1 | X_i,\hat{\theta})$
\item The derivatives are a little bit tricky, but the idea is the same.
\item This is what STATA should be doing when you type: \tt{mfx, compute}
\end{itemize}
\end{frame}

\begin{frame}{Delta Method: Other Examples}
Often we have a regression like:
\begin{eqnarray*}
log Y_i = \beta_0 + \beta_1 X_i + \gamma Income_i + \epsilon_i
\end{eqnarray*}
And we are interested in $\beta_1 / \gamma$ so that we have $\beta_i$ in units of ``dollars''. Again Delta Method Works fine here.\\
\end{frame}


\begin{frame}{Delta Method: Some Failures}
But we need to be careful.  Suppose that $\theta \approx 0$ and 
\begin{itemize}
\item $h(x)  = |X|$
\item $h(x)  = 1/X$
\item $h(x)  = \sqrt{X}$
\end{itemize}
These situations can arise in practice when we have weak instruments or other problems.
\end{frame}



\end{document}