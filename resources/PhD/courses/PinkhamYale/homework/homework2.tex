\documentclass[11pt, a4paper]{article}
%\usepackage{geometry}
\usepackage[inner=1.5cm,outer=1.5cm,top=2.5cm,bottom=2.5cm]{geometry}
\pagestyle{empty}
\usepackage{graphicx}

\usepackage[usenames,dvipsnames]{color}
\definecolor{darkblue}{rgb}{0,0,.6}
\definecolor{darkred}{rgb}{.7,0,0}
\definecolor{darkgreen}{rgb}{0,.6,0}
\definecolor{red}{rgb}{.98,0,0}
\usepackage[colorlinks,pagebackref,pdfusetitle,urlcolor=darkblue,citecolor=darkblue,linkcolor=darkred,bookmarksnumbered,plainpages=false]{hyperref}
\renewcommand{\thefootnote}{\fnsymbol{footnote}}



\thispagestyle{plain}

%%%%%%%%%%%% LISTING %%%
\usepackage{listings}
\usepackage{caption}
\DeclareCaptionFont{white}{\color{white}}
\DeclareCaptionFormat{listing}{\colorbox{gray}{\parbox{\textwidth}{#1#2#3}}}
\captionsetup[lstlisting]{format=listing,labelfont=white,textfont=white}
\usepackage{verbatim} % used to display code
\usepackage{fancyvrb}
\usepackage{acronym}
\usepackage{amsthm}
\VerbatimFootnotes % Required, otherwise verbatim does not work in footnotes!



\definecolor{OliveGreen}{cmyk}{0.64,0,0.95,0.40}
\definecolor{CadetBlue}{cmyk}{0.62,0.57,0.23,0}
\definecolor{lightlightgray}{gray}{0.93}



\lstset{
%language=bash,                          % Code langugage
basicstyle=\ttfamily,                   % Code font, Examples: \footnotesize, \ttfamily
keywordstyle=\color{OliveGreen},        % Keywords font ('*' = uppercase)
commentstyle=\color{gray},              % Comments font
numbers=left,                           % Line nums position
numberstyle=\tiny,                      % Line-numbers fonts
stepnumber=1,                           % Step between two line-numbers
numbersep=5pt,                          % How far are line-numbers from code
backgroundcolor=\color{lightlightgray}, % Choose background color
frame=none,                             % A frame around the code
tabsize=2,                              % Default tab size
captionpos=t,                           % Caption-position = bottom
breaklines=true,                        % Automatic line breaking?
breakatwhitespace=false,                % Automatic breaks only at whitespace?
showspaces=false,                       % Dont make spaces visible
showtabs=false,                         % Dont make tabls visible
columns=flexible,                       % Column format
morekeywords={__global__, __device__},  % CUDA specific keywords
}

%%%%%%%%%%%%%%%%%%%%%%%%%%%%%%%%%%%%
\begin{document}
\begin{center}
  {\Large \textsc{Problem Set 2}}

  MGMT 737
\end{center}
\begin{center}
  Spring 2024
\end{center}
%\date{September 26, 2014}

\begin{enumerate}
\item Propensity Scores. This analysis will use the dataset
  from Problem 1 as well as the PSID dataset from Dehijia and
  Wahba, \texttt{lalonde\_psid.csv}. These datasets have identical
  variables. The new dataset is a sample of observations from the
  Panel Survey of Income Dynamics that can be used as alternative
  control observations. Importantly, these observations were not in
  the initial randomization.
  \begin{enumerate}
  \item Using \texttt{age}, \texttt{education}, \texttt{hispanic},
    \texttt{black}, \texttt{married}, \texttt{nodegree}, \texttt{RE74}
    and \texttt{RE75}, construct a propensity score using the
    \emph{treated} group in \texttt{lalonde\_nsw.csv} and the control
    sample of \texttt{lalonde\_psid.csv}.  Use a logit regression
    model to do so (you may use a canned routine to run the
    regression). Report the average p-score for the treated and
    control samples, and plot the propensity score densities for the
    treatment and control groups.
  \item Using your p-score estimates, calculate the IPW and SIPW
    estimate for control and treated mean of the outcome, and the
    average treatment effect. Contrast these estimates to the control
    mean of the outcome from the NSW sample, and the treatment effect
    from last week's problem set.
  \item Compare the ATE in the previous question to the treatment
    effect estimated using a linear regression using the PSID and NSW
    treatment sample, with \texttt{age}, \texttt{education},
    \texttt{hispanic}, \texttt{black}, \texttt{married},
    \texttt{nodegree}, \texttt{RE74} and \texttt{RE75} as controls.
  \item Now revisit your estimates from part a and b, and following
    Crump et al. (2009), discard all units with estimated propensities
    outside the range of $[0.1, 0.9]$. Reestimate the IPW and SIPW
    estimator of the ATE from part b using this trimmed sample.
  \item Finally, calculate the IPW and SIPW estimates for the ATE
    using this trimmed sample for Black and non-Black
    individuals. Compare this estimate to the ATE for Black and
    non-Black individuals using the full randomized sample.
    \item We now consider what our results look like if we just consider difference-in-differences. Estimate a linear regression model (using canned software package like \texttt{lm} in R is fine -- be sure to cluster on the individual) with two specifications:
    \begin{enumerate}
      \item The difference in \texttt{RE78} less \texttt{RE75} on the left hand side against a treatment indicator. 
      \item \texttt{RE78} on the left hand side against a treatment indicator and \texttt{RE75}.
    \end{enumerate}
  Report the estimates and compare to your previous estimates. Can you think of other specifications that might work? 
\end{enumerate}
\end{enumerate}

\end{document}